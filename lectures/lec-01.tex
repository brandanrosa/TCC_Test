\lesson{}{14.1 - Matrix Addition and Scalar Multiplicaton}

\subsection{Complex Numbers}
\label{sub_sec:sub_section_1}

\large
A complex number looks like this:
    $$
        a + bi
    $$

It has a real part, $a$, and an imaginary part, $bi$.
    
    \vspace{\stretch{1}}

    \begin{definition}[The Imaginary Unit: $i$ ]
    
        \begin{align*} 
        i^2 &= -1 \\ 
        i &= \sqrt{-1}
        \end{align*}

    \end{definition}

    \vspace{\stretch{1}}

%\begin{theorem}
%This is a theorem.
%\end{theorem}
%\begin{proof}
%This is a proof.
%\end{proof}

    \begin{example}
     $2 + 3i$ , $-5 - 6i$ , and  $\pi - i$  are all Complex Numbers.
    \end{example}
    
    \vspace{\stretch{1}}    
    
    \begin{example}
    Numbers like $3i$  and $-7i$ are called Pure Imaginary Numbers.
    \end{example}

    \vspace{\stretch{1}}
    
    \begin{note}
    We define the square root of any negative number as:
     $$ \sqrt{-a} = i\sqrt{a} $$
    \end{note} 
    
    \vspace{\stretch{1}}
    
%\begin{explanation}
%This is an explanation.
%\end{explanation}
%\begin{claim}
%This is a claim.
%\end{claim}
%\begin{corollary}
%This is a corollary.
%\end{corollary}
%\begin{prop}
%This is a proposition.
%\end{prop}
%\begin{lemma}
%This is a lemma.
%\end{lemma}

    \begin{example}
        Simplify
            \begin{align*}
                 \sqrt{-25} &= 5i \\ 
                 \sqrt{-72} &= 6i \sqrt{2} \\
                 -5 \sqrt{-9} &= -15i 
            \end{align*}
    \end{example}

    \vspace{\stretch{1}}

\newpage
%--------------------------------------------------------------------------

%\begin{solution}
%\end{solution}

%\begin{exercise}
%This is an exercise.
%\end{exercise}

%\begin{definition}[Definition]
%This is a definition.
%\end{definition}

%\begin{note}
%This is a note.
%\end{note}

% subsection sub_section_1 (end)

%--------------------------------------------------------------------------

\subsection{Addition \& Subtraction}
\label{sub_sec:sub_section_2}

    \begin{definition}[Adding Complex Numbers]
        $$ (a + bi) + (c + di) = (a + c) + (b + d)i $$
    \end{definition}

    \vspace{\stretch{1}}

    \begin{definition}[Subtracting Complex Numbers]
        $$ (a + bi) - (c + di) = (a - c) + (b - d)i $$
    \end{definition}

    \vspace{\stretch{1}}

    \begin{example}
        $$ (8 - i) + (5 + 4i) = 13 + 3i $$
    \end{example}

    \vspace{\stretch{1}}

    \begin{example}
        $$ (7 - 6i) - (3 - 6i) = 4 $$
    \end{example}

    \vspace{\stretch{1}}

    \begin{example}
        $$ 13 - (2 + 7i) + 5i = 11 - 2i $$
    \end{example}

    \vspace{\stretch{1}}
%------------------------------------------------------------------

\subsection{Multiplication}
\label{sub_sec:sub_section_2}

    \begin{example}
            \begin{align*}
                 4i (6 - 3i) &= 24i - 12i^2 \\
                             &= 12 + 24i 
            \end{align*}
    \end{example}

    \vspace{\stretch{1}}

    \begin{example}
            \begin{align*}
                 (2 + 3i)(4 + 5i) &= 8 + 10i + 12i + 15i^2 \\
                                  &= 8 + 22i - 15 \\
                                  &= -7 + 22i
            \end{align*}
    \end{example}

    \vspace{\stretch{1}}

\newpage
%------------------------------------------------------------------

\subsection{Solving Quadratics Revisited}
\label{sub_sec:sub_section_2}

    \begin{note}
    Fundamental Theorem of Algebra states that every polynomial equation of degree $n$ has $n$ roots (solutions, zeros, x-intercepts) in the complex numbers.
    \end{note}
    
    \vspace{\stretch{1}}

    \begin{example}
        Solve
        $$ f(x) = x^2 + 4 $$
        Replace $f(x)$ with $0$, 
            \begin{align*}
              x^2 + 4 &= 0  \\
                  x^2 &= -4 \\
                    x &= \pm \sqrt{-4} \\
                      &= \pm 2i
            \end{align*}
    \end{example}

    \vspace{\stretch{1}}

    \begin{example}
        Solve
        $$ g(x) = (x + 2)^2 + 32 $$
        Replace $g(x)$ with $0$, 
            \begin{align*}
              (x + 2)^2 + 32 &= 0  \\
                   (x + 2)^2 &= -32 \\
                       x + 2 &= \pm \sqrt{-32} \\
                       x + 2 &= \pm 4i \sqrt{2} \\
                             &= -2 \pm 4i \sqrt{2}
            \end{align*}
    \end{example}

    \vspace{\stretch{1}}

\newpage






